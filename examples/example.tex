% -pd(docx) -pdf
\documentclass{report}
\usepackage{amsmath, graphicx}
\begin{document}



\chapter{Normal Equations}

It should be emphasized that for systems in which the inputs are known ahead of time and in which there are no disturbances it is advisable to use open-loop control.  Closed-loop control systems have advantages only when unpredictable disturbance and/or unpredictable variations in system components are present. Note that the  output power rating partially determines the cost,weight,and size of a control system.  The number of components used in a closed-loop control system is more than that for  a corresponding open-loop control system. Thus, the closed-loop control system is  generally higher in cost and power.To decrease the required power of a system,open-  loop control may be used where applicable.A proper combination of open-loop and  closed-loop controls is usually less expensive and will give satisfactory overall system  performance

\begin{equation}
\alpha=R_{C} + \frac{\beta_{}}{\theta} \, \psi
\end{equation}

\begin{align}
\begin{split}
\alpha &= R_{C} + \frac{\beta_{}}{\theta} \cdot \psi\\
 &= 5 \cdot R_{c}\\
 &= 69\\
\end{split}
\end{align}

It should be emphasized that for systems in which the inputs are known ahead of time and in which there are no disturbances it is advisable to use open-loop control.  Closed-loop control systems have advantages only when unpredictable disturbance and/or unpredictable variations in system components are present. Note that the  output power rating partially determines the cost,weight,and size of a control system.  The number of components used in a closed-loop control system is more than that for  a corresponding open-loop control system. Thus, the closed-loop control system is  generally higher in cost and power.To decrease the required power of a system,open-  loop control may be used where applicable.A proper combination of open-loop and  closed-loop controls is usually less expensive and will give satisfactory overall system  performance.

\begin{equation}
\alpha = \frac{\beta}{\theta}\times\psi+R_C
\end{equation}

\begin{align}
\begin{split}
\alpha &= \frac{\beta}{\theta}\times\psi+R_C\\
&=\sin{\left(5\times x_R\right)} + 6\mathrm{m}\\
&=56\mathrm{m}\\
\end{split}
\end{align}

\chapter{Normal Aserar}
\section{Without Units}
$F = ma$ It should be emphasized that for systems in which the inputs are known ahead of time and in which there are no disturbances it is advisable to use open-loop control.  Closed-loop control systems have advantages only when unpredictable disturbance  and/or unpredictable variations in system components are present. Note that the  output power rating partially determines the cost,weight,and size of a control system.  The number of components used in a closed-loop control system is more than that for  a corresponding open-loop control system. Thus, the closed-loop control system is generally higher in cost and power.To decrease the required power of a system,open-  loop control may be used where applicable.A proper combination of open-loop and  closed-loop controls is usually less expensive and will give satisfactory overall system  performance.

$x = 5$ 

\begin{align}
\begin{split}
y	&= x^{2} + 9 \cdot x\\
	&= 5^{2} + 9 \times 5\\
	&= 70\\
\end{split}
\end{align}

\section{With Units}

$F = ma$ It should be emphasized that for systems in which the inputs are known ahead of time and in which there are no disturbances it is advisable to use open-loop control.  Closed-loop control systems have advantages only when unpredictable disturbance  and/or unpredictable variations in system components are present. Note that the  output power rating partially determines the cost,weight,and size of a control system.  The number of components used in a closed-loop control system is more than that for  a corresponding open-loop control system. Thus, the closed-loop control system is generally higher in cost and power.To decrease the required power of a system,open-  loop control may be used where applicable.A proper combination of open-loop and  closed-loop controls is usually less expensive and will give satisfactory overall system  performance.

$\beta_{} = 58\,\mathrm{in}$ 

\begin{align}
\begin{split}
\zeta	&= \beta_{} + 69 \cdot \mathrm{m}\\
		&= 58\,\mathrm{in} + 69 \times 1\,\mathrm{m}\\
		&= 70.473\,\mathrm{m}\\
\end{split}
\end{align}

\chapter{Aserar With Arrays}
\section{Without Units}

$F = ma$ It should be emphasized that for systems in which the inputs are known ahead of time and in which there are no disturbances it is advisable to use open-loop control.  Closed-loop control systems have advantages only when unpredictable disturbance  and/or unpredictable variations in system components are present. Note that the  output power rating partially determines the cost,weight,and size of a control system.  The number of components used in a closed-loop control system is more than that for  a corresponding open-loop control system. Thus, the closed-loop control system is generally higher in cost and power.To decrease the required power of a system,open-  loop control may be used where applicable.A proper combination of open-loop and  closed-loop controls is usually less expensive and will give satisfactory overall system  performance.

\begin{align}
\begin{split}
D_{f}	&= \operatorname{linspace}{\left (1,5,9 \right )}\\
		&= \operatorname{linspace}{\left (1,5,9 \right )}\\
		&= \left[\begin{matrix}1\\1.5\\\vdots\\5\end{matrix}\right]\\
\end{split}
\end{align}

\begin{align}
\begin{split}
S_{d}	&= 98 \cdot D_{f} + 5\\
		&= 98 \times \left[\begin{matrix}1\\1.5\\\vdots\\5\end{matrix}\right] + 5\\
		&= \left[\begin{matrix}103\\152\\\vdots\\495\end{matrix}\right]\\
\end{split}
\end{align}

\section{With Units}
$F = ma$ It should be emphasized that for systems in which the inputs are known ahead of time and in which there are no disturbances it is advisable to use open-loop control.  Closed-loop control systems have advantages only when unpredictable disturbance  and/or unpredictable variations in system components are present. Note that the  output power rating partially determines the cost,weight,and size of a control system.  The number of components used in a closed-loop control system is more than that for  a corresponding open-loop control system. Thus, the closed-loop control system is generally higher in cost and power.To decrease the required power of a system,open-  loop control may be used where applicable.A proper combination of open-loop and  closed-loop controls is usually less expensive and will give satisfactory overall system  performance.

\begin{align}
\begin{split}
R_{D}	&= \mathrm{m} \cdot \operatorname{linspace}{\left (1,5,32 \right )}\\
		&= 1\,\mathrm{m} \times \operatorname{linspace}{\left (1,5,32 \right )}\\
		&= \left[\begin{matrix}1\\1.129\\\vdots\\5\end{matrix}\right]\,\mathrm{m}\\
\end{split}
\end{align}

\begin{align}
\begin{split}
T_{m}	&= 56 \cdot \mathrm{s}\\
		&= 56 \times 1\,\mathrm{s}\\
		&= 56\,\mathrm{s}\\
\end{split}
\end{align}

\begin{align}
\begin{split}
V^{D}_{y}	&= \frac{R_{D}^{2}}{T_{m}}\\
			&= \frac{\left(\left[\begin{matrix}1\\1.129\\\vdots\\5\end{matrix}\right]\,\mathrm{m}\right)^{2}}{56\,\mathrm{s}}\\
			&= \left[\begin{matrix}1.79(10^{-02})\\2.28(10^{-02})\\\vdots\\0.446\end{matrix}\right]\,\mathrm{{m^{2}}\slash{s}}\\
\end{split}
\end{align}

\chapter{Aserar With Matrices}
\section{Without Units}

$F = ma$ It should be emphasized that for systems in which the inputs are known ahead of time and in which there are no disturbances it is advisable to use open-loop control.  Closed-loop control systems have advantages only when unpredictable disturbance  and/or unpredictable variations in system components are present. Note that the  output power rating partially determines the cost,weight,and size of a control system.  The number of components used in a closed-loop control system is more than that for  a corresponding open-loop control system. Thus, the closed-loop control system is generally higher in cost and power.To decrease the required power of a system,open-  loop control may be used where applicable.A proper combination of open-loop and  closed-loop controls is usually less expensive and will give satisfactory overall system  performance.

$M_{T} = \left[\begin{matrix}1 & 4 1 & \cdots & 2\\4 1 & 5 1 & \cdots & 3\\\vdots & \vdots & \ddots & \vdots\\1 & 4 & \cdots & 4\end{matrix}\right]\,\mathrm{m}$ 

\begin{align}
\begin{split}
M_{a}	&= 2 \cdot M_{T}\\
		&= 2 \times \left[\begin{matrix}1 & 4 1 & \cdots & 2\\4 1 & 5 1 & \cdots & 3\\\vdots & \vdots & \ddots & \vdots\\1 & 4 & \cdots & 4\end{matrix}\right]\,\mathrm{m}\\
		&= \left[\begin{matrix}2 & 8 & \cdots & 4\\8 & 10 & \cdots & 6\\\vdots & \vdots & \ddots & \vdots\\2 & 8 & \cdots & 8\end{matrix}\right]\,\mathrm{m}\\
\end{split}
\end{align}

\section{With Units}

$F = ma$ It should be emphasized that for systems in which the inputs are known ahead of time and in which there are no disturbances it is advisable to use open-loop control.  Closed-loop control systems have advantages only when unpredictable disturbance  and/or unpredictable variations in system components are present. Note that the  output power rating partially determines the cost,weight,and size of a control system.  The number of components used in a closed-loop control system is more than that for  a corresponding open-loop control system. Thus, the closed-loop control system is generally higher in cost and power.To decrease the required power of a system,open-  loop control may be used where applicable.A proper combination of open-loop and  closed-loop controls is usually less expensive and will give satisfactory overall system  performance.

\end{document}
